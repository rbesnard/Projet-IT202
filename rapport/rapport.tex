\documentclass[12pt]{article}

\usepackage{graphicx}
\usepackage[french]{babel}
\usepackage[utf8]{inputenc}

%%%%%%%%%%%%%%%% Lengths %%%%%%%%%%%%%%%%
\setlength{\textwidth}{15.5cm}
\setlength{\evensidemargin}{0.5cm}
\setlength{\oddsidemargin}{0.5cm}

%%%%%%%%%%%%%%%% Variables %%%%%%%%%%%%%%%%

\def\titre{Ordonnanceur de threads}
\def\equipe{Romain BESNARD\\ Amandine GEORGES\\ Lun JIANG\\ Quentin LAMBERT}


\begin{document}

%%%%%%%%%%%%%%%% Header %%%%%%%%%%%%%%%%
\fbox{
  \noindent\begin{minipage}{0.98\textwidth}
  \vskip 0mm
  \noindent
      { \begin{tabular}{p{7.5cm}}
          {\bfseries \sffamily
            Projet de système d'exploitation} \\
          {\itshape \titre}
      \end{tabular}}
      \hfill
      \fbox{\begin{tabular}{l}
          {~\hfill \bfseries \sffamily Equipe :
            \hfill~} \\[2mm]
          \equipe \\

      \end{tabular}}
      \vskip 4mm ~

      ~~~\parbox{0.95\textwidth}{\small \textit{Résumé~:} \sffamily Le
        projet qui nous a été proposé consiste en l'implémentation
        d'un ordonnanceur de threads. Nous étions tenus dans un
        premier temps d'exécuter un programme exemple manipulant
        certaines primitives du module.}

      \vskip 1mm ~

  \end{minipage}
}

~~\\
~~\\
\section {La structure thread}

Un type \verb'thread_t' devait être implémenté, nous avons décidé
qu'il devrait représenter un pointeur vers une structure thread que
nous allons maintenant détailler.\\
% detail et justification de la structure thread
\begin{verbatim}
struct thread {
    ucontext_t uc;
    void *retval;
    struct GList * sleeping_list;
}
\end{verbatim}
Le champ \verb'uc' sert à conserver le contexte d'exécution du thread,
\verb'retval' quant à lui permet de stocker la valeur de retour du
thread quand celui-ci est dans la liste "zombie". Mais quand il
est dans la liste "ready", \verb'retval' peut contenir la valeur de retour du
thread qu'il attend. Le dernier champ est la liste des
threads dormants dans l'attente de la fin de l'exécution de ce thread.

\section {Les threads dans l'état "ready" et "zombie"}

Nous avons choisi de stocker ces threads dans deux listes distinctes
pour chacun des états. Les threads "ready" sont ceux prêts à
s'exécuter, et les "zombies" ont terminé leur exécution et attendent
que leur valeur de retour soit récupérée.\\
Nous avons pris comme convention que la tête de la liste "ready" est
le thread en cours d'exécution.

\section {Utilisation de la GLib}

Pour nous abstraire de l'implementation du TAD list nous avons utilisé
la GLib, nous fournissant les fonctions nécessaires à la manipulation
des listes de threads tout en nous assurant qu'il n'y aura aucune
fuite mémoire.

\section {L'implémentation des fonctions}

\begin{verbatim}
thread_t thread_self(void);
\end{verbatim}
Avec notre choix de mettre le thread courant en tête de la liste "ready", 
cette fonction renvoie simplement la tête de la liste "ready".
~~\\
\begin{verbatim}
int thread_create(thread_t *newthread, void *(*func)(void *), void *funcarg);
\end{verbatim}


Cette fonction a la responsablité d'ajouter le thread main dans la
liste "ready", si c'est la première fois qu'on crée un thread.  Comme
la liste "ready" est initialement vide, la fonction vérifie si cette
liste est vide, et si c'est le cas, elle alloue une instance de la
structure thread, sauvegarde le contexte du main dans cette instance
et la place à la tête de la liste.  Puis elle crée une instance de la
structure thread avec la fonction passée en paramètre et retourne un
thread\_t pointant sur cette structure.  ~~\\
\begin{verbatim}
int thread_yield(void);
\end{verbatim}
Cette fonction fait un $swapcontext$ sur le contexte de la tête de la
liste "ready" (qui est le thread courant) et celui de l'élément
suivant.  ~~\\
\begin{verbatim}
int thread_join(thread_t thread, void **retval);
\end{verbatim}
Cette fonction vérifie tout d'abord si le thread qu'on veut attendre
est dans la liste "ready". Si c'est le cas alors elle se met dans la
liste des dormants de ce thread et passe la main au suivant.  Sinon
elle vérifie si ce thread est dans la liste "zombie", auquel cas 
elle prend la valeur de retour de ce thread et continue son exécution.
Enfin si ce thread n'est pas dans la liste "zombie" alors elle
retourne -1 pour dire que l'appel a échoué et que ce thread n'existe
pas.  ~~\\
\begin{verbatim}
void thread_exit(void *retval);
\end{verbatim}
Cette fonction réveille tous les threads de sa liste des dormants et
met à jour son champ retval avec la valeur en paramètre. Puis elle se
met dans la liste "zombie" et passe la main au suivant.

\section{La mémoire}
Pour l'instant aucune libération de mémoire n'est faite. En
particulier nous ne nous sommes pas encore posé la question de quand
libérer les threads de la liste zombie. Cela pose un problème
évident lorsque le nombre de threads devient important. Nous nous
attellerons donc en priorité à la résolution de ce problème.

\section{Améliorations prévue}
Nous avons implémenté les programmes parallèles du calcul de la suite
de Fibonacci et de la somme des élément d'un tableau par
diviser-pour-régner. Cela nous permettra de tester les performances de
notre bibliothèque, en évaluant le temps d'exécution en fonction de la
taille de l'entrée. Il nous semble que la gestion du thread principal
est déjà prise en compte, il conviendra de s'en assurer. Nous pensons
mettre au point des fonctions de synchronisation de type
sémaphore. Nous aimerions aussi nous pencher sur la gestion de la
préemption et des signaux.


\end{document}
